% !TeX spellcheck = en_US
% !TeX encoding = UTF-8
%\documentclass[a4paper,headsepline,12pt,bibliography=totoc]{scrreprt}
\documentclass[a4paper,headsepline,12pt]{scrartcl}
\usepackage[T1]{fontenc}
\usepackage[utf8]{inputenc}
\usepackage[english]{babel}
\usepackage[hyphens,spaces,obeyspaces]{url}
\usepackage[pdfborder={0 0 0}]{hyperref}
\usepackage[backend=bibtex,style=numeric-comp]{biblatex}
\usepackage[babel,style=english,english=american]{csquotes}
%\usepackage[singlespacing]{setspace}
\usepackage{lmodern}
\usepackage[usenames,dvipsnames,table]{xcolor}
%\usepackage{cite}
\usepackage[binary-units]{siunitx}
\usepackage{float}
\usepackage{setspace}
\usepackage{mathcomp}
\usepackage{amsmath}
\usepackage[font=small]{caption} 
\usepackage[font=footnotesize]{subcaption}
%\usepackage{ae}
\usepackage{caption}
\usepackage[final]{graphicx}
\usepackage{listings}
\usepackage{enumitem}
\usepackage{adjustbox}
\usepackage{tikz}
\usepackage{xspace}
\usepackage{amssymb}
%\usepackage{showframe}
%\usepackage[prependcaption,textsize=tiny,colorinlistoftodos]{todonotes}
\usepackage[disable]{todonotes}

\lstset{
	%numbers=left,
	breaklines=true,
	tabsize=4,
	basicstyle=\ttfamily,
	commentstyle=\color{red},
}

% alle floats zentrieren
\makeatletter
\g@addto@macro\@floatboxreset\centering
\makeatother

\newcommand{\eg}{e.\,g.\xspace}


\title{AIR - Homework 6}
\date{\today}
\author{Maximilian Mensing\\Torsten Jandt}


\begin{document}
\maketitle{}
\section{Uniform-Cost Search}
Uniform-cost search (\emph{UCS}), breadth first search (\emph{BFS}) and depth first search (\emph{DFS}) require no heuristics. In UCS nodes are only distinguished by \( g(n)\). This means, when expanding to a new node, UCS expands to the node, which has the lowest path cost so far.
\paragraph{BFS is a Case of UCS}
BFS is a special version of UCS. In case the expansion of any node costs the same, UCS behaves like BFS.
Uniform Cost search is an 

\paragraph{UCS, DFS and BFS are Special Cases of Greedy Best-First Search}
BFS is a case of Greedy Best-First Search, when the cost to reach any node in the next layer is higher than the cost to reach a nodes in the same layer. This way the algorithm traverses the search tree layer wise. 

Since greedy search algorithms do not use backtracking, the first node in the next layer, which is reachable from a current node, has to be reachable at the lowest path cost. In this case, DFS and  Greedy Best-First Search would behave alike until the goal has been reached. 

\paragraph{UCS is a Case of A*}
In general, UCS is the uninformed version A*. While UCS has no information at any point how close it is to the goal, this information if provided to A* by heuristics.
%UCS considers all reachable nodes, in order to keep the total path cost at a minimum, whereas A* expands always the path with the lowest cost \( f(n)\). Whenever 

\newpage
\section{A*}
\paragraph{Completeness of A*} A search algorithm is complete, if it can tell whether the goal is reachable or not. If it is reachable it is able to find this path. If the algorithms gets stuck in a loop it is not complete. To avoid loops, A* needs an admissible heuristic, like the manhattan distance. 
\paragraph{Termination of A*} A* ends the search, once the goal has been found, or it has been discovered tat the goal is unreachable
%https://ocw.mit.edu/courses/aeronautics-and-astronautics/16-410-principles-of-autonomy-and-decision-making-fall-2010/lecture-notes/MIT16_410F10_lec04.pdf
\paragraph{A* with Consistent Heuristics}
If a consistent heuristic is used with the A* algorithm, then A* expands nodes in order of increasing \(f(n)\) values.



\end{document}